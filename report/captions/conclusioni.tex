\section{Conclusione}\label{sec:conclusione}
    Abbiamo esaminato l'interazione tra la diffusione di un'epidemia e la percezione del rischio nelle reti multiplex,
    utilizzando approssimazioni di campo medio e un metodo auto-organizzato.
    Questo metodo determina la soglia di percolazione in una sola simulazione.
    La nostra attenzione si è concentrata sulle reti multiplex, poiché le persone si infettano attraverso contatti
    fisici, ma spesso raccolgono informazioni da una rete informativa, che può differire notevolmente dalle reti fisiche.
    Abbiamo scoperto che la somiglianza tra le reti fisiche e informative è fondamentale per fermare l'infezione se il
    livello di precauzione è sufficientemente alto.
    Se le reti sono molto diverse, non è possibile evitare l'epidemia.
    Questa transizione avviene bruscamente, soprattutto per bassi valori di probabilità di infezione, senza segnali
    premonitori.
    Questo fenomeno rappresenta un monito contro l'eccessiva dipendenza da Internet come unica fonte di informazione.
    Nonostante il mondo virtuale permetta una rapida diffusione delle informazioni, le epidemie reali si propagano
    ancora nel mondo fisico.
    Questo è particolarmente rilevante per le malattie che colpiscono gruppi marginalizzati o etnici, che non ricevono
    sufficiente copertura mediatica.
    Infine, nella società attuale, è possibile che individui di diversi livelli sociali stabiliscano un contatto fisico,
    e quindi la possibilità di contagio, anche se non sono in contatto a livello informativo.