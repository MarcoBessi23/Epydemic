\section{Conclusione}\label{sec:conclusione}
    Abbiamo esaminato l'interazione tra la diffusione di un'epidemia e la percezione del rischio nelle reti multiplex,
    utilizzando approssimazioni di campo medio e un metodo self-organized che permette di determinare la soglia
    di percolazione in una sola simulazione.
    La nostra attenzione si è concentrata sulle reti multiplex, poiché le persone si infettano attraverso contatti
    fisici, ma spesso raccolgono informazioni da una rete informativa, che può differire notevolmente dalle reti fisiche.
    Abbiamo scoperto che la somiglianza tra le reti fisiche e informative è fondamentale per fermare l'infezione se il
    livello di precauzione è sufficientemente alto.
    Se le reti sono molto diverse, non è possibile evitare l'epidemia.
    Questa transizione avviene bruscamente, soprattutto per bassi valori di probabilità di infezione, senza segnali
    premonitori.
    Un fenomeno simile può essere preso come esempio degli svantaggi dell'eccessiva dipendenza da fonti esterne,
    quali i social media, come unica fonte di informazione.