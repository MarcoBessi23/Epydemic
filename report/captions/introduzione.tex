\section{Introduzione}\label{sec:introduzione}
    Il documento discute la diffusione delle epidemie e l'importanza della percezione del rischio nelle società umane.
    Analizza le pandemie del passato, come l'influenza spagnola e l'influenza asiatica, e come queste hanno influenzato
    la risposta delle organizzazioni sanitarie pubbliche.
    Il documento sottolinea come le informazioni false o esagerate sulle epidemie possano creare paura e panico.

    Inoltre, il documento esplora l'importanza delle reti nella diffusione delle malattie, sottolineando come le reti
    profondamente connesse possano accelerare la diffusione di una malattia.
    Viene discusso un modello che tiene conto della percezione del rischio e delle precauzioni che le persone prendono
    quando diventano consapevoli di un'epidemia.

    Infine, il documento esamina l'importanza delle reti di informazione, sia fisiche che virtuali, nella diffusione
    delle malattie.
    Viene proposto un modello di rete multiplex, che tiene conto sia delle reti di contatti fisici che delle reti di
    informazione, per studiare l'interazione tra la percezione del rischio e la diffusione della malattia.