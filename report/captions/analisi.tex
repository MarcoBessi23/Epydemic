\section{Analisi del Modello}\label{sec:analisi-del-modello}

Per prima cosa descriviamo i meccanismi per generare reti regolari, casuali e scale-free.
Denotiamo con $A$ la matrice di adiacenza della rete dove $a_{ij} = 1$ se c'è un arco da
$j$ a $i$ e $a_{ij} = 0$ altrimenti.
Denotiamo con $k_i = \sum a_{ij}$ la connettività del sito $i$ e 
con $j^{(i)}_{1}\cdots j^{(i)}_{k_{i}}$ i suoi vicini $\left\{ j : a_{ij}=1 \right\}$.

Consideriamo solo reti simmetriche con $N$ nodi e $2mN$ collegamenti, in modo che la connettività media 
$\left\langle k \right\rangle$di ciascun nodo sia pari a $2m$.

\begin{itemize}
    \item \textbf{Reti regolare Unidimensionale:} I nodi sono disposti ad anello.
        Ogni dato nodo stabilisce un collegamento con gli $m$ nodi più vicini alla sua destra.
    \item \textbf{Reti casuali:} Ogni nodo stabilisce $m$ collegamenti con nodi scelti casualmente, evitando auto-loop e collegamenti multipli.
        La distribuzione di probabilità delle reti casuali è quella di Poisson, $P(k) = \frac{z^k e^{-z}}{k!}$ dove $z =\left\langle k \right\rangle$.
    \item \textbf{Reti scale-free:} Utilizziamo un modello fissando anche un cutoff $K$.
        Prima, a ciascun nodo $i$ viene assegnata una connettività $k_i$ estratta da una distribuzione di legge di potenza $P(k) = A k^{-\gamma}$, $m \leq k \leq K$ con $A=(\gamma -1)/(m^{1-\gamma} -K^{1-\gamma})$.
        I rimanenti archi sono aggiunti in modo casuale, evitando auto-loop e connessioni multiple, infine il numero totale di archi viene tagliato per ottenere il grado medio desiderato.
\end{itemize}

Siamo interessati a reti multiplex, ottenute dalla combinazione di uno strato fisico e uno strato virtuale.
Per cominciare generiamo le due reti scegliendo tra le tre tipologie sopra descritte.

La rete di informazione viene poi costruita come segue:
\begin{itemize}
    \item \textbf{Rete fisica:}
        \begin{itemize}
            \item Per ogni nodo $i$ della rete fisica considera il suo vicinato $\mathcal{N}(i)$, per ogni $j\in \mathcal{N}(i)$
                aggiungi un arco con direzione da $i$ verso $j$ con probabilità $1-q$.
        \end{itemize}
    \item \textbf{Rete virtuale:}
        \begin{itemize}
            \item Esegui la stessa procedura con la rete virtuale ma con probabilità $q$.
        \end{itemize}
\end{itemize}

In quanto questo procedimento si ripete in modo indipendente per ogni nodo, la rete di informazioni risultante non
conserva più la sua simmetria originaria (un determinato collegamento può essere scelto da uno dei suoi vertici ma non dall'altro).
Il parametro $q$ indica il peso assegnato alla rete virtuale: maggiore è il valore di $q$ verso $1$, maggiore sarà la
somiglianza tra la rete delle informazioni e quella virtuale.
In pratica, questo parametro regola quanto gli individui attribuiscono importanza alle loro relazioni virtuali.
La quantità $Nqk$ rappresenta la differenza tra la rete di informazioni e quella fisica.

Questa procedura ci permette di studiare gli effetti su un'epidemia della differenza tra la rete fisica, dove si verifica la
diffusione e quella di informazione, dove gli attori diventano consapevoli della malattia,
cioè su cui valutano il rischio di essere infettati.

\subsection{Mean-field approximation}\label{subsec:mean-field-approximation}
% FIXME: aggiungere riferimenti a teoria, aggiungere roba sul campo medio? e controllare quanto scritto ora
Per prima cosa il valore critico di $J$ viene ricercato utilizzando i metodi della teoria di campo medio e sempre assumendo
che tutti i nodi abbiano la stessa connettività.
Abbiamo usato quindi la seguente formula di aggiornamento della percentuale $c$ di infetti sul grafo basata
sui metodi di approssimazioni di campo medio:

\begin{equation}
    c(t+1)=\sum_{s=0}^{k}\binom{k}{s}c(t)^{s}(1-c(t))^{k-s}s\tau\exp\left(-J\frac{s}{k}\right)\label{eq:mean-field}
\end{equation}

A questo punto fissato un certo valore di $\tau$, abbiamo preso il minimo livello di precauzione $J$ tale per cui $c$ dopo un
certo numero di passi diventa nullo, ovvero tale da azzerare il numero di infetti. 
Nella teoria il valore critico di $J$ viene fornito dalla seguente funzione:

\begin{equation}
    J_{c}=k\ln(k\tau)\label{eq:mean-field-critical}
\end{equation}

Viene ricavata considerando lo stato stazionario $c(t+1)=c(t)$ con $c\to0$ ed eseguendo alcune manipolazioni algebriche.

Nella fase di test e di valutazione andremo a verificare se i valori predetti dalla formula sono vicini a quelli
che si osservano simulando l'infezione.

\subsection{Self-organized percolation}\label{subsec:self-organized-percolation}
Ci siamo poi dedicati allo studio della diffusione dell'epidemia su grafi mediante l'impiego di modelli basati sulla percolazione.
Abbiamo confrontato i valori teorici con quelli derivati dalle simulazioni,
al fine di ottenere una visione completa e confrontabile dei risultati.

\subsubsection{Directed percolation}\label{subsubsec:directed-percolation}
Per cominciare abbiamo considerato grafi in cui i nodi non hanno percezione del rischio, l'unica determinante 
dell'infezione in questo scenario è la bare infection probability $\tau$.
Ogni nodo ha un proprio valore di $\tau$ ed è possibile dimostrare che il $\tau$ ottimale per il nodo $i$ al tempo $t+1$ è dato da:
\begin{equation}
    \tau_{i}(t+1) = \min_{j= j^{(i)}_{1},\dots, j^{(i)}_{k_{i}}}\max(r_{ij}(t), \tau_{i}(t))\label{eq:directed-percolation}
\end{equation}
con $k_{i}$ grado del nodo $i$ e $j$ indice dei vicini.
Dato un certo tempo sufficientemente grande ma finito $T$, il valore: $\min_{i=1,\dots,n}\tau_{i}(T)$ rappresenta
la soglia di percolazione al di sopra della quale c'è almeno un sito infetto nel reticolo.
Andremo a verificare anche in questo caso se i valori predetti dalla formula sono molto vicini a quelli che si osservano
simulando l'infezione.

\subsubsection{Infezione con la percezione del rischio}\label{subsubsec:infezione-con-la-percezione-del-rischio}
In questa sezione analizziamo un problema più complesso, cioè assumiamo che i nodi della rete siano dotati di
un valore $J$ che indica la percezione del rischio di infezione.
Anche in questo scenario è possibile ricavare una successione di valori ottimali di $J$ come segue:
\begin{equation}
    J_{i}(t+1) = \max_{j= j^{(i)}_{1},\dots,j^{(i)}_{k_{i}}}\min\left(J_{j}(t),\frac{-k_{ i}}{s_{i}(J_{j}(t))}\ln\left(\frac{r_{ij}(t)}{\tau}\right)\right)\label{eq:risk-perception}
\end{equation}

In modo simile a quanto fatto in precedenza prendiamo: $\max_{i= 1,\dots,n}J_{i}(T)$ come valore critico.

\subsubsection{Multiplex networks}\label{subsubsec:multiplex-networks}
In conclusione abbiamo affrontato il problema della percolazione su reti Multiplex, come nel caso precedente 
l'infezione avviene sul network fisico ma la
percezione del rischio è influenzata dalla topologia della rete di informazione e in particolare dal numero di
infetti su quest'ultima.
La formula che si ricava è molto simile a quella precedente:
\begin{equation}
    J_{i}(t+1)=\max_{j^{(i)}_{1},\dots,j^{(i)}_{k_{i}}}\min(J_{j}(t),\frac{-k_{ i}}{\overline{s_{i}}(J_{j}(t))}\ln(\frac{r_{ij}(t)}{\tau}))\label{eq:multiplex}
\end{equation}
l'unica differenza è nel termine $\overline{s_{i}}$ che rappresenta il numero di vicini infetti percepito
dal nodo $i$ e viene calcolato sul grafo informativo.
Quello che ci si può aspettare è che al crescere del valore di q la rete informativa si avvicini sempre più a quella
virtuale, influenzando molto la percezione del rischio per un dato nodo che concentra la sua attenzione su nodi che non
gli sono vicini realmente, sottovalutando il reale livello di rischio del mondo fisico.
