\section{Analisi del Modello}\label{sec:analisi-del-modello}

Per prima cosa descriviamo i meccanismi per generare reti regolari, casuali e scale-free.
Denotiamo con $A$ la matrice di adiacenza della rete dove $a_{ij} = 1$ se c'è un arco da
$j$ a $i$ e $a_{ij} = 0$ altrimenti.
Denotiamo con $k_i = \sum a_{ij}$ la connettività del sito $i$ e 
con $j^{(i)}_{1}\cdots j^{(i)}_{k_{i}}$ i suoi vicini $\left\{ j : a_{ij}=1 \right\}$. 
Consideriamo solo reti simmetriche con $N$ nodi e $2mN$ collegamenti, in modo che la connettività media 
$\left\langle k \right\rangle$ di ciascun nodo sia pari a $2m$.

(i) Regolare unidimensionale: I nodi sono disposti ad anello.
Ogni dato nodo stabilisce un collegamento con gli $m$ nodi più vicini alla sua destra.

(ii) Casuale: Ogni nodo stabilisce $m$ collegamenti con nodi scelti casualmente, evitando auto-loop e collegamenti
multipli.
La distribuzione di probabilità delle reti casuali è quella di Poisson, $P(k) = \frac{z^k e^{-z}}{k!}$ 
dove $z =\left\langle k \right\rangle$.

(iii) Scale-free: Utilizziamo un modello fissando anche un cutoff $K$.

Prima, a ciascun nodo $i$ viene assegnata una connettività $k_i$ estratta da una distribuzione 
di legge di potenza $P(k) = A k^{-\gamma}$, $m \leq k \leq K$ con $A=(\gamma -1)/(m^{1-\gamma} -K^{1-\gamma})$.
I rimanenti archi sono aggiunti in modo casuale, evitando auto-loop e connessioni multiple, infine
il numero totale di archi viene "potato" per ottenere il grado medio desiderato.
Questo meccanismo ci permette di generare reti scale-free con un dato esponente $\gamma$.

Siamo interessati a reti multiplex,ottenute dalla combinazione di uno strato fisico e uno strato virtuale.
Per cominciare generiamo le due reti scegliendo tra (i),(ii) e (iii). 
La rete di informazione viene poi costruita come segue:

(i) Per ogni nodo $i$ della rete fisica considera il suo vicinato $\mathcal{N}(i)$, per ogni $j\in \mathcal{N}(i)$
aggiungi un arco con direzione da $i$ verso $j$ con probabilità $1-q$.

(ii) Esegui la stessa procedura con la rete virtuale ma con probabilità $q$.

Poiché questo processo viene ripetuto indipendentemente per ogni nodo, la rete di informazione risultante 
non è più simmetrica (un dato collegamento può essere scelto da uno dei suoi vertici ma non dall'altro). 
Il valore $q$ rappresenta il peso che viene attribuito alla rete virtuale, più $q$ è vicino a $1$ e più la 
rete delle informazioni somiglia a quella virtuale. Nella pratica questo parametro regola quanta 
importanza gli individui danno alle loro relazioni virtuali.
La quantità $Nqk$ corrisponde alla differenza tra la rete di informazione e la rete fisica.

Questa procedura ci permette di studiare gli effetti su un epidemia della differenza tra la rete fisica, dove si verifica la
diffusione e quella di informazione, dove gli attori diventano consapevoli della malattia,
cioè su cui valutano il rischio di essere infettati.

\subsection{Mean-field approximation}\label{subsec:mean-field-approximation}
In questa sezione abbiamo usato la seguente formula di aggiornamento della percentale $c$ di infetti sul grafo basata
sui metodi di approsimazioni di campo medio $c(t+1)=\sum_{s=0}^{k}\binom{k}{s}c^{s}(1-c)^{k-s}s\tau\exp(-J\frac{s}{k})$.
a questo punto fissato un certo valore di $\tau$, abbiamo preso il minimo livello di precauzione $J$ tale per cui $c$ dopo un 
certo numero di passi diventa nullo, ovvero tale da azzerare il numero di infetti. 
Nella teoria il valore critico di $J$ viene fornito dalla seguent funzione $J_{c}=k\ln(k\tau)$, ricavata 
considerando lo stato stazionario $c(t+1)=c(t)$ con $c\to0$ ed eseguendo alcune manipolazioni algebriche.
Come possiamo osservare nel seguito i valori predetti dalla teoria combaciano con quelli restituiti simulando 
il processo infettivo.
\subsection{Self-organized percolation}\label{subsec:self-organized-percolation}
Ci siamo poi interessati a studiare la diffusione dell'epidemia su grafi tramite modelli basati 
sulla percolazione, confrontando anche in questo caso i valori teorici con quelli ottenuti dalle simulazioni.

\subsubsection{Directed percolation}\label{subsubsec:directed-percolation}
Per cominciare abbiamo considerato grafi in cui i nodi non hanno percezione del rischio, l'unica determinante 
dell'infezione in questo scenario è la bare infection probability $tau$. Ogni nodo ha un proprio valore di $\tau$
ed è possibile dimostrare che il $\tau$ ottimale per il nodo $i$ al tempo $t+1$ è dato da $\tau_{i}(t+1) = \min_{j= j^{(i)}_{1},..j^{(i)}_{k_{i}}}\max(r_{ij}(t), \tau_{i}(t))$
con $k_{i}$ grado del nodo $i$ e $j$ indice dei vicini.
Dato un certo tempo sufficientemente grande ma finito $T$, il valore  $\min_{i=1,..,n}\tau_{i}(T)$ 
rappresenta la soglia di percolazione al di sopra della quale c'è almeno un sito infetto nel reticolo.
Anche in questo caso siamo riusciti ad ottenere che i valori predetti dalla formula sono molto vicini a 
quelli che si osservano simulando l'infezione.
\subsubsection{Infezione con la percezione del rischio}\label{subsubsec:infezione-con-la-percezione-del-rischio}
In seguito abbiamo reso il problema più complesso, assumendo che i nodi della rete siano dotati di 
un valore $J$ che indica la percezione del rischio di infezione, anche in questo scenario è possibile 
ricavare una successione di valori ottimali di $J$ come segue $J_{i}(t+1)=\max_{j^{(i)}_{1},...,j^{(i)}_{k_{i}}}min(J_{j}(t),\frac{-k_{i}}{s_{i}(J_{j}(t))}\ln(\frac{r_{ij}(t)}{\tau}))$ 
e in modo simile a quanto fatto in precedenza prendere $\max_{i= 1,..,n}J_{i}(T)$ come valore critico.
In questo caso è interessante notare come al crescere di $tau$ oltre la soglia critica ci sia sempre un 
$J$ per cui è possibile arrestare la diffusione dell'epidemia.
\subsubsection{Multiplex networks}\label{subsubsec:multiplex-networks}
Per finire abbiamo affrontato il problema della percolazione in un network fisico come sopra dove però la 
percezione del rischio dipende dalla topologia della rete di informazione e in particolare dal numero di
infetti su quest'ultima.
La formula che si ricava è molto simile a quella precedente, $J_{i}(t+1)=\max_{j^{(i)}_{1},...,j^{(i)}_{k_{i}}}min(J_{j}(t),\frac{-k_{i}}{\overline{s}_{i}(J_{j}(t))}\ln(\frac{r_{ij}(t)}{\tau}))$ 
l'unica differenza è nel termine $\overline{s}_{i}$ che rappresenta il numero di vicini infetti percepito
dal nodo $i$ e viene calcolato sul grafo informativo. Quello che ci si aspetta è che al crescere del valore di q
la rete informativa si avvicini sempre più a quella virtuale, questo può influenzare molto la percezione 
del rischio per un dato nodo che concentra la sua attenzione su nodi che non gli sono vicini realmente, 
sottovalutando il reale livello di rischio del mondo fisico. Quello che si osserva dai diagrammi di fase è
proprio che fissato un certo valore $\tau$ sufficientemente alto c'è sempre una soglia critica di $q$ 
oltre la quale diventa praticamente impossibile fermare l'epidemia.