\section{Abstract}\label{sec:abstract}
    L'introduzione del documento esamina la storia delle pandemie, mettendo in evidenza come eventi passati come
    l'influenza spagnola e l'influenza asiatica abbiano avuto un impatto significativo sulla società.
    Sottolinea l'importanza della percezione del rischio nella gestione delle epidemie e come le informazioni false o
    esagerate possano influenzare la risposta del pubblico.
    Il documento introduce anche l'idea di utilizzare le reti, sia fisiche che virtuali, per studiare la diffusione
    delle malattie e modellare l'effetto delle informazioni virtuali sulla consapevolezza del pubblico.
    Infine, propone un modello di rete multiplex per studiare l'interazione tra la percezione del rischio e la
    diffusione della malattia.