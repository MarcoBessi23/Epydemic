\section{Risultati}\label{sec:risultati}

\subsection{Mean-field approximation}\label{subsec:res-mean-field-approximation}
    L'approssimazione di campo medio è utile per studiare il comportamento del sistema in condizioni di 
    equilibrio, ma non tiene conto dell'effetto delle fluttuazioni locali.
    Come possiamo vedere dalla figura ~\ref{fig:mf_critical_j}, il calcolo del campo medio approssima bene il valore
    critico di $J$, sia nel caso teorico che in quello pratico, confermando la validità teorica del metodo.

    \begin{figure}[h]
        \includegraphics[width=\textwidth]{mf_critical_j}\caption{Mean-field approximation}
        \label{fig:mf_critical_j}
    \end{figure}

\subsection{Simple percolation}\label{subsec:res-simple-percolation}
    In questo caso abbiamo fatto varie simulazioni in base al numero di nodi e al valore di connettività $k$.
    I risultati mostrati in ~\ref{fig:prob_percolation} e ~\ref{fig:prob_percolation_2} mostrano i risultati ottenuti.

    \begin{figure}[h]
        \begin{minipage}{0.5\textwidth}
            \includegraphics[width=\linewidth]{critical_t}\label{fig:prob_percolation}
        \end{minipage}
        \begin{minipage}{0.5\textwidth}
            \includegraphics[width=\linewidth]{critical_t}\label{fig:prob_percolation_2}
        \end{minipage}
        \caption{Valutazione della probabilità di percolazione}
    \end{figure}

\subsection{Infezione percezione del rischio}\label{subsec:res-infezione-con-la-percezione-del-rischio}
    Il modello di infezione con la percezione del rischio tiene conto del fatto che le persone possono adottare misure
    precauzioni per proteggersi da una malattia diversamente dal caso precedente.
    Questo metodo è utile per studiare l'effetto della percezione del rischio sulla diffusione di una malattia, ma non
    tiene conto dell'effetto delle informazioni virtuali. Nel nostro esperimento abbiamo verificato che 
    sotto le nostre assunzioni sui grafi, i valori restituiti dal metodo di percolazione si avvicinano molto
    a quelli ottenuti dalla teoria di campo medio, suggerendo che in modo approssimativo il sistema goda della 
    proprietà di self organized criticality. Abbiamo quindi un modo automatico per ricavare il livello critico
    della percezione del rischio senza dover fare un tuning dei parametri di controllo attraverso numerose
    simulazioni.
    \begin{figure}[h]
        \includegraphics[width=\textwidth]{graficoJC}\caption{Confronto self percolation and simulation}
        \label{fig:graficoJC}
    \end{figure}

\subsection{Multiplex networks}\label{subsec:res-multiplex-networks}
    I risultati ottenuti dalla simulazione del modello multiplex sono mostrati in figura ~\ref{fig:diagram_phase}.
    Possiamo vedere come all'aumentare di $t$ e $q$ l'infezione si diffonde e non si è in grado di fermarla, infatti i
    valori di $Jc$ aumentano notevolmente.
    Nella visualizzazione del diagramma di fase, possiamo vedere come la curva della soglia di percolazione approssimi 
    un'iperbole.

    \begin{figure}[h]
        \includegraphics[width=\textwidth]{q_plot}\label{fig:diagram_phase}
        \caption{Diagramma di fase al variare di q e $\tau$}
    \end{figure}
